
% ---------------------------------------------------------------------------
% Author guideline and sample document for EG publication using LaTeX2e input
% D.Fellner, v1.20, Jan 18, 2023

\documentclass{egpubl}
\usepackage{STAG2025}

% --- for  Annual CONFERENCE
% \ConferenceSubmission   % uncomment for Conference submission
% \ConferencePaper        % uncomment for (final) Conference Paper
% \STAR                   % uncomment for STAR contribution
% \Tutorial               % uncomment for Tutorial contribution
% \ShortPresentation      % uncomment for (final) Short Conference Presentation
% \Areas                  % uncomment for Areas contribution
% \Education              % uncomment for Education contribution
% \Poster                 % uncomment for Poster contribution
% \DC                     % uncomment for Doctoral Consortium
%
% --- for  CGF Journal
% \JournalSubmission    % uncomment for submission to Computer Graphics Forum
% \JournalPaper         % uncomment for final version of Journal Paper
%
% --- for  CGF Journal: special issue
% \SpecialIssueSubmission    % uncomment for submission to , special issue
% \SpecialIssuePaper         % uncomment for final version of Computer Graphics Forum, special issue
%                          % EuroVis, SGP, Rendering, PG
% --- for  EG Workshop Proceedings
% \WsSubmission      % uncomment for submission to EG Workshop
\WsPaper           % uncomment for final version of EG Workshop contribution
% \WsSubmissionJoint % for joint events, for example ICAT-EGVE
% \WsPaperJoint      % for joint events, for example ICAT-EGVE
% \Expressive        % for SBIM, CAe, NPAR
% \DigitalHeritagePaper
% \PaperL2P          % for events EG only asks for License to Publish

% --- for EuroVis 
% for full papers use \SpecialIssuePaper
% \STAREurovis   % for EuroVis additional material 
% \EuroVisPoster % for EuroVis additional material 
% \EuroVisShort  % for EuroVis additional material
% \MedicalPrize  % uncomment for Medical Prize (Dirk Bartz) contribution, since 2021 part of EuroVis

% Licences: for CGF Journal (EG conf. full papers and STARs, EuroVis conf. full papers and STARs, SR, SGP, PG)
% please choose the correct license
%\CGFStandardLicense
%\CGFccby
%\CGFccbync
%\CGFccbyncnd

% !! *please* don't change anything above
% !! unless you REALLY know what you are doing
% ------------------------------------------------------------------------
\usepackage[T1]{fontenc}
\usepackage{dfadobe}  

%\usepackage{cite}  % comment out for biblatex with backend=biber 
% ---------------------------
%\biberVersion
\BibtexOrBiblatex
%\usepackage[backend=biber,bibstyle=EG,citestyle=alphabetic,backref=true]{biblatex} 
%\addbibresource{egbibsample.bib}
% ---------------------------  
\electronicVersion
\PrintedOrElectronic

% for including postscript figures
% mind: package option 'draft' will replace PS figure by a filename within a frame
\ifpdf \usepackage[pdftex]{graphicx} \pdfcompresslevel=9
\else \usepackage[dvips]{graphicx} \fi

\usepackage{egweblnk} 
% end of prologue

% ====== OUR DEFS ==========
\usepackage[english]{babel}
\usepackage{amsfonts}
\usepackage{amsmath}
%\usepackage{amsthm}
\usepackage{mathtools}
\usepackage{libertinus}
\usepackage{algpseudocodex}
\usepackage{algorithm}
\usepackage{csquotes}
\usepackage{hyperref}
\usepackage{listings}
\usepackage{xcolor}
\usepackage{float}
\usepackage[capitalize,noabbrev]{cleveref}
\usepackage{tikz-3dplot}
\usepackage{multirow}

% It seems that libertinus already loads the inconsolata font, so adding the package for that results in a conflict. Specifying the correct font family is enough.\usepackage{natbib}

\renewcommand{\ttdefault}{zi4}

\DeclareMathOperator*{\argmax}{arg\,max}
\DeclareMathOperator*{\argmin}{arg\,min}

\renewcommand{\algorithmicrequire}{\textbf{Input:}}
\renewcommand{\algorithmicensure}{\textbf{Output:}}

\newcommand{\True}{\textsc{true}}
\newcommand{\False}{\textsc{false}}
\newcommand{\EValid}{\textsc{valid}}
\newcommand{\EInvalid}{\textsc{invalid}}
\newcommand{\EUncertain}{\textsc{uncertain}}

\newcommand{\reals}{\mathbb{R}}
\newcommand{\ints}{\mathbb{Z}}
\newcommand{\nats}{\mathbb{N}}
\newcommand{\floor}[1]{\lfloor#1\rfloor}
\newcommand{\ceil}[1]{\lceil#1\rceil}

\newcommand{\intervals}{\mathbb{I}}
\newcommand{\support}{\operatorname{supp}}
\newcommand{\timedep}[1]{\overline{#1}}
\newcommand{\intlo}[1]{\underline{#1}}
\newcommand{\inthi}[1]{\overline{#1}}
\newcommand{\inclusion}[1]{\Box #1}

\newcommand{\arcsinh}{\operatorname{arcsinh}}
\newcommand{\arccosh}{\operatorname{arccosh}}
\newcommand{\arctanh}{\operatorname{arctanh}}
\newcommand{\arctantwo}{\operatorname{arctan2}}
\newcommand{\erf}{\operatorname{erf}}
\newcommand{\erfc}{\operatorname{erfc}}

\newcommand{\solve}{\textsc{Solve}}
\newcommand{\minimize}{\textsc{Minimize}}
%\newcommand{\keyword}[1]{\texttt{#1}}

\newtheorem{theorem}{Theorem}
\newtheorem{lemma}{Lemma}
\newtheorem{definition}{Definition}

%------------------------------------------------------------------------ % Alter some LaTeX defaults for better treatment of figures: 
% See p.105 of "TeX Unbound" for suggested values. 
% See pp. 199-200 of Lamport's "LaTeX" book for details. % General parameters, for ALL pages:
\renewcommand{\topfraction}{0.9} % max fraction of floats at top 
\renewcommand{\bottomfraction}{0.8} % max fraction of floats at bottom 
% Parameters for TEXT pages (not float pages): 
\setcounter{topnumber}{4} 
\setcounter{bottomnumber}{4} 
\setcounter{totalnumber}{8} % 2 may work better 
\setcounter{dbltopnumber}{4} % for 2-column pages 
\renewcommand{\dbltopfraction}{0.95} % fit big float above 2-col. text 
\renewcommand{\textfraction}{0.07} % allow minimal text w. figs % Parameters for FLOAT pages (not text pages): 
\renewcommand{\floatpagefraction}{0.7} % require fuller float pages % N.B.: floatpagefraction MUST be less than topfraction !! 
\renewcommand{\dblfloatpagefraction}{0.7} % require fuller float pages %------------------------------------------------------------------------

% CONFIGURATION OF listings
\definecolor{codegreen}{rgb}{0,0.6,0}
\definecolor{codegray}{rgb}{0.5,0.5,0.5}
\definecolor{codepurple}{rgb}{0.58,0,0.82}
\definecolor{backcolour}{rgb}{0.95,0.95,0.92}

\newcommand{\FS}[1]{\textcolor{teal}{\textbf{FS}: #1}}
%\newcommand{\ZH}[1]{\textcolor{blue}{\textbf{ZH}: #1}}
\newcommand{\MA}[1]{\textcolor{orange}{\textbf{MA}: #1}}
%\newcommand{\DZ}[1]{\textcolor{green}{\textbf{DZ}: #1}}
\newcommand{\EP}[1]{\textcolor{purple}{\textbf{EP}: #1}}
%\newcommand{\DP}[1]{\textcolor{red}{\textbf{DP}: #1}}

\lstdefinestyle{smallcode}{
    backgroundcolor=\color{backcolour}, 
    commentstyle=\color{codegreen},
    keywordstyle=\color{magenta},
    keywordstyle=[3]\color{red},
    numberstyle=\tiny\color{codegray},
    stringstyle=\color{codepurple},
    basicstyle=\ttfamily\tiny,
    breakatwhitespace=false,
    breaklines=true,
    captionpos=b,
    numbers=left,
    numbersep=5pt,
    tabsize=2,
}

\lstdefinestyle{largecode}{
    style=smallcode,
    basicstyle=\ttfamily\scriptsize,
    numbers=none,
}

\lstdefinelanguage{miso}{
    language=Python, % inherit Python syntax
    deletekeywords=[2]{sum},
    morekeywords={with, as, assert},
    morekeywords=[3]{variables, arguments, poly_space, geo_map, bases, collapse, expand, subdiv_strategy, generate, Context, vector, basis, jacobian, det, sum, prod},
}


\lstset{style=smallcode}
% END listings


\title[Short title here]%
      {\emph{TIGHT} Intervals for Provably Correct Geometric Computation}

% for anonymous conference submission please enter your SUBMISSION ID
% instead of the author's name (and leave the affiliation blank) !!
% for final version: please provide your *own* ORCID in the brackets following \orcid; see https://orcid.org/ for more details.
\author[Anonymous]
{\parbox{\textwidth}{\centering Anonymous authors}}
%{\parbox{\textwidth}{\centering D.\,W. Fellner\thanks{Chairman Eurographics Publications Board}$^{1,2}$\orcid{0000-0001-7756-0901}
 %       and S. Behnke$^{2}$\orcid{0000-0001-5923-423X} 
%        S. Spencer$^2$\thanks{Chairman Siggraph Publications Board}
%        }
%       \\
% For Computer Graphics Forum: Please use the abbreviation of your first name.
%{\parbox{\textwidth}{\centering $^1$TU Darmstadt \& Fraunhofer IGD, Germany\\
 %        $^2$Graz University of Technology, Institute of Computer Graphics and Knowledge Visualization, Austria
%        $^2$ Another Department to illustrate the use in papers from authors
%             with different affiliations
%       }
%}
%}
% ------------------------------------------------------------------------

% if the Editors-in-Chief have given you the data, you may uncomment
% the following five lines and insert it here
%
% \volume{36}   % the volume in which the issue will be published;
% \issue{1}     % the issue number of the publication
% \pStartPage{1}      % set starting page


%-------------------------------------------------------------------------
\begin{document}

\teaser{
 \includegraphics[width=0.4\linewidth]{fig/512x512.png}
 \centering
  \caption{this is the teaser}
\label{fig:teaser}
}

\maketitle
%-------------------------------------------------------------------------
\begin{abstract}
Interval arithmetic is a practical method for robust computation, bridging the gap between inexact floating-point arithmetic and slow, exact arithmetic (such as rational or arbitrary-precision). In this system, numbers are represented as intervals bounded by floating-point numbers. Operations are performed conservatively, guaranteeing that the resulting interval contains the exact mathematical result.

We present a new C++ library for interval arithmetic that supports both algebraic and transcendental functions. A key feature of this library is its use of correctly rounded operations, ensuring the resulting interval is the smallest floating-point interval that contains the true result. We demonstrate the library’s effectiveness by applying it to complex non-polynomial problems, including continuous collision detection for geometric primitives undergoing roto-translational motion.
\EP{Check "correctly rounded". Forse detto così è misleading.}
%-------------------------------------------------------------------------
%  ACM CCS 1998
%  (see https://www.acm.org/publications/computing-classification-system/1998)
% \begin{classification} % according to https://www.acm.org/publications/computing-classification-system/1998
% \CCScat{Computer Graphics}{I.3.3}{Picture/Image Generation}{Line and curve generation}
% \end{classification}
%-------------------------------------------------------------------------
%  ACM CCS 2012
%  (see https://www.acm.org/publications/class-2012)
%The tool at \url{http://dl.acm.org/ccs.cfm} can be used to generate
% CCS codes.
%Example:
\begin{CCSXML}
<ccs2012>
<concept>
<concept_id>10010147.10010371.10010352.10010381</concept_id>
<concept_desc>Computing methodologies~Collision detection</concept_desc>
<concept_significance>300</concept_significance>
</concept>
<concept>
<concept_id>10010583.10010588.10010559</concept_id>
<concept_desc>Hardware~Sensors and actuators</concept_desc>
<concept_significance>300</concept_significance>
</concept>
<concept>
<concept_id>10010583.10010584.10010587</concept_id>
<concept_desc>Hardware~PCB design and layout</concept_desc>
<concept_significance>100</concept_significance>
</concept>
</ccs2012>
\end{CCSXML}

\ccsdesc[300]{Computing methodologies~Collision detection}
\ccsdesc[300]{Hardware~Sensors and actuators}
\ccsdesc[100]{Hardware~PCB design and layout}


\printccsdesc   
\end{abstract}  
%-------------------------------------------------------------------------

% !TEX root = STAG25-MiSo2.tex

\section{Introduction}
\label{sec:introduction}
Robust geometric computation is a critical component of many applications, such as collision detection, minimum distance computation, element inversion, and Boolean operations \cite{something}. Although floating-point arithmetic is fast, it can easily generate inaccurate results that may lead to unpredictable, often catastrophic outcomes, especially in simulations \cite{something}. In contrast, exact computation, using methods like rational arithmetic or arbitrary precision floating-point arithmetic, is extremely slow and often impractical.
Moreover, such methods are exact only for algebraic operations. 

Interval arithmetic bridges this gap by providing a viable alternative. 
It offers a balance between performance and accuracy, giving conservative estimates of exact computations at a moderate speed penalty compared to standard floating-point arithmetic. In this model, all real values are represented as intervals bounded by floating-point numbers. Expressions on these intervals are computed in a way that guarantees the resulting interval will contain the true mathematical result.
While the width of an interval can grow during computation, it is crucial to minimize this growth to maintain precision. This is achieved through \emph{correctly rounded} operations, where each elementary operation returns the tightest possible floating-point interval that contains the exact result.
%More fundamentally, if elementary computations are not correctly rounded, the same expression %evaluated with the same sequence of operations
% may give different results on different architectures, or even on the same architecture once the underlying mathematical library is changed. 

The IEEE 754 standard prescribes correctly rounded results only for the algebraic operations and the square root; all other 
%transcendental
operations are just \emph{recommended} to be correctly rounded, but there is no guarantee, depending on their implementation.
%The CORE-MATH project \cite{core-math} provides correctly rounded implementations for all the most common transcendental operations. 
Consequently, nowadays no libraries exist that guarantee the creation of as-tight-as-possible intervals when the expressions involve this kind of operations.
See Section \ref{sec:related} for further discussion.

% TOLTA FRASE SEGUENTE, RIPETE LE PRIMA DELL'INTRO
%Since such a kind of expressions are useful in diverse graphics applications (e.g., non-linear element inversion, collision detection, Boolean operations), i
In this paper, we describe the design principles of our TIGHT library for interval arithmetic, which always produces as-tight-as-possible intervals, is faster than any existing library, and supports transcendental functions.
%
Our original contributions include:
\begin{enumerate}
\item We extend the NFG library \cite{nfg} -- which provides the most efficient implementation of interval arithmetic to date, but is limited to algebraic operations -- with transcendental operations, based on the CORE-MATH floating-point correctly rounded implementation \cite{Sibidanov2022}.
In Section \ref{sec:functions}, we discuss the challenges involved in extending transcendental operators to intervals while guaranteeing correct rounding. 
\item We integrate our library with the recent Domain Specific Language MiSo \cite{Sichetti2025} that supports the fast prototyping of non-linear constraint solving and optimization. 
Extension of the language with transcendental functions largely broadens its spectrum of applicability. 
\item We demonstrate the effectiveness and efficiency of our library by implementing surface-surface intersection between non-algebraic surfaces, and continuous collision detection between geometric primitives undergoing roto-translational motion.  We also compare our library against the Filib library \cite{filib}, achieving a faster performance.
\end{enumerate}


%THE REST IS FROM SIGGRAPH PAPER - PROBABLY OBSOLETE, MAYBE SOMETHING USEFUL FOR THE RELATED

%Non-linear constraint solving is fundamental to graphics and scientific computing, with applications ranging from collision detection and minimal distance computation to element inversion and Boolean operations. A vast literature addresses this topic (Section \ref{sec:related}). For example, \cite{RTR4,Akenine2024} summarize methods for static collision detection between proxies, referencing over 100 algorithms tailored to different primitive pairs and accuracy/efficiency trade-offs.
%Similar per-primitive-pair specialization is required for minimal distance queries and, likewise, each finite element (FE) type and order necessitates custom code for positive Jacobian checks.
%This complexity doubles when considering time-dependent scenarios.

%While real-time applications often restrict primitives to boxes due to limited computational resources, high-fidelity simulations may require conservative, high-accuracy predicates \cite{snyder92}.
%Testing the correctness and ensuring the efficient, accurate implementation of these algorithms is a major challenge \cite{Wang2021}. The difficulty of generalizing theoretical improvements across different cases hinders progress in this pervasive and crucial family of algorithms, essential to modern computing.

%In contrast to algorithm specialization, Snyder \cite{snyder92} proposed a general framework, based on interval analysis, for conservative solutions to high-order constrained optimization. This framework offers two algorithms: \solve, which finds all solutions to a non-linear constraint system, and \minimize, which finds the constrained global minimum of a function. For \solve, the conservative algorithm returns a region guaranteed to contain all solutions (if any), potentially including points near the feasible domain. For \minimize, it returns a value less than or equal to the true minimum and within a bounded distance of it. In both cases, this conservativeness accounts for numerical rounding errors.

%Although often considered slower than methods like Newton's minimization, recent work \cite{Wang2021,Chen2024} 
 %demonstrates the effectiveness and relative efficiency of this conservative approach, particularly when seeking guaranteed solutions. 

%Snyder's approach uses Natural Interval Extensions (NIE) to compute \emph{inclusion functions} (\cref{sub:inclusion}) that bound function ranges over domains, by composition of interval operators (\cref{app:interval}). Although general, NIE's convergence to the true range via domain decomposition can be slow. 
%For the common case of polynomials, tighter bounds are achievable via their Bézier representation \cite{Lengagne:2020,stahl_interval_1995,johnen2013}. 
%However, Bézier representation can be computationally expensive for polynomials with many terms. 

%We employ a hybrid approach, blending Bézier inclusion functions and NIE. Decomposing polynomial expressions into simpler forms (fewer variables or lower degree) allows us to construct a spectrum of inclusion functions that ranges from fully NIE-based (expanded expressions) to fully Bézier-based (collapsed expressions).
%Hybrid solutions can dramatically improve efficiency. For example, our hybrid solver for continuous collision detection between high-order polynomial patches is orders of magnitude faster than purely NIE-based and purely Bézier-based solutions (\cref{sec:results}).

%We developed MiSo on top of such a hybrid approach. MiSo is a Python-based domain-specific language (DSL) for the specification of \solve\ and \minimize\ problems. MiSo enables the user to quickly explore possible hybrid approaches by changing a few lines of code.
%From a simple specification, the MiSo compiler produces a numerically robust C++ solver for the given problem, automatically generating all the necessary representations of the functions involved, the related transformations required for domain subdivision, and the evaluation of inclusion functions.

%Domain decomposition and interval arithmetic are used to guarantee conservative results. Setting a compile-time flag switches to a faster, non-conservative computation mode based on standard floating-point arithmetic.
%A known limitation of subdivision-based methods is that they suffer from a curse of dimensionality; hence, our method may become impractical for problems in many dimensions. However, we show that we are able to achieve competitive performance for a number of fundamental geometric problems, especially those involving high-order geometry.

%We demonstrate competitive performance against hand-optimized code for key computer graphics problems, including linear and high-order continuous collision detection, and finite element validity checks.

%MiSo is available as an open-source project at \url{https://gitlab.com/fsichetti/miso}.


%% !TEX root = STAG25-MiSo2.tex

\section{Background and state of the art}
\label{sec:related}

Interval arithmetic \cite{hickey2001} provides a set of operations on real intervals $\intervals$ such that if $x\in a=[\intlo{a},\inthi{a}] \in \intervals$ and $y\in b=[\intlo{b},\inthi{b}] \in \intervals$,
then $x \star y  \in a \star b$, where $\star$ in the right-hand side is the interval version of operation $\star$ on reals.
We are interested in intervals whose lower and upper bounds can be represented by FP numbers.

From now on, the set of representable FP numbers is denoted by $\mathbb{F}$. When $a$ and $b$ are in $\mathbb{F}$, the result $r = a \star b$ may be not in $\mathbb{F}$, and hence not representable. In that case $i=[fp^{-}(r),fp^{+}(r)]$, where $fp^{-}(r) = max(f : f \in \mathbb{F}, f \leq r)$ and $fp^{+}(r) = min(f : f \in \mathbb{F})$, is the tightest representable interval containing $r$.
IEEE 754 requires that when $\star$ is an algebraic operation (or the square root) an implementation of $\star$ must round the theoretically exact result $r$ to either $fp^{-}(r)$ or $fp^{+}(r)$, depending on the current \emph{rounding mode}.
In most modern architectures the rounding mode is controlled by a particular register within the CPU, and specific system functions exist to set it. Therefore, a trivial approach to create a tight interval for the operation $a \star b$ is to (1) set the rounding mode to $towards -\infty$, (2) execute $a \star b$ to determine the interval's lower bound, (3) set the rounding mode to $towards +\infty$, (4) execute $a \star b$ to determine the interval's upper bound.
Since setting the rounding mode is typically slower than executing arithmetic operations, a more efficient approach is (1) set the rounding mode to $towards +\infty$, (2) execute $(-a) \star b$ and switch the sign of the result to determine the interval's lower bound, (3) execute $a \star b$ to determine the interval's upper bound. Furthermore, if no other parts of the program require a different rounding, step (1) can be executed only once at the beginning.
This approach is used by existing interval arithmetic libraries such as Boost \cite{bronnimann2006} and CGAL \cite{cgal}.

Another possibility is to deconstruct the binary representation of the result $r$ to directly modify the mantissa, exponent and sign, and produce a reasonably small interval around $r$. This approach is used by Filib and Filib++ \cite{filib} \FS{Filib++ has multiple modes, is this true for all modes? should we be more specific?}.
Alternatively, the error propagation can be analyzed to derive a bound $\epsilon$ on the rounding so that the interval $i=[r-\epsilon,r+\epsilon]$ is guaranteed to contain $r$. This is how libraries such as BIAS \cite{bias} or GAOL \cite{gaol} work.

The aforementioned existing libraries were comprehensively compared by Tang and colleagues \cite{tang2022} who evaluated diverse aspects, including their correctness, efficiency and precision (in terms of interval tightness). Their conclusion is that only filib and filib++ are always correct when transcendental functions are involved, although the intervals they produce might be larger than necessary. In contrast, Boost and BIAS may produce intervals that do not contain the exact result. Also, Tang's evaluation could verify that libraries that use the rounding mode produce tighter intervals.

With the exception of rather old architectures, most existing CPUs provide SIMD registers and instructions that proved useful to accelerate interval arithmetic libraries \cite{lambov2008}. Here the basic idea is to store both the bounds in a single 128bit-wide register and perform operations on both bounds in parallel. This and other optimizations exploiting more recent AVX architectures were  included in the NFG library \cite{nfg} that, to the best of our knowledge, represents the fastest existing library at the time of writing. Since NFG exploits the rounding mode, it is also guaranteed to produce as-tight-as-possible intervals for all the algebraic operations and the square root. Our TIGHT library wraps around NFG while adding many other elementary and transcendental functions while keeping the guarantee to produce tight intervals.

\subsection{Correct rounding}
\emph{Correct rounding} refers to the property that an implementation
of a mathematical function $f$ has if, for any $x$ that is representable and contained in the domain of $f$, it returns the same results one would get by rounding the exact result $f(x)$ to the target representation.
While this property may seem obvious in theory, this is not what happens in real life: implementations of mathematical functions often consists of several steps, and the composition of two CR functions is not CR in general.
The IEEE754 standard for floating point arithmetic requires a compliant implementation of a function to round correctly for all inputs \cite{ieee}. Indeed, required operations such as summation, subtraction, multiplication, division, and square roots produce the same, correctly rounded results on any IEEE754-compliant machine.
However, this is not true for \emph{recommended} functions like $\sin$ or $\log$: because they are not mandatory, mathematical libraries are allowed to implement fast, non-CR routines that are not IEEE754-conforming, but the language implementation as a whole will be conforming as long as the mandatory operations are CR.

%% !TEX root = STAG25-MiSo2.tex

\section{Implementing elementary functions}
\label{sec:functions}
In order to implement robust, conservative and tight interval extensions of elementary functions, we rely on existing \emph{correctly rounded} implementations of functions.
The CORE-MATH Project \cite{Sibidanov2022} provides a collection of fast, correctly rounded implementations of most commonly used mathematical functions.
For ease of use, we created a minimal C++ library of double-precision CORE-MATH implementations that is publicly available at \url{}.

The functions that we implemented are:
\begin{itemize}
	\item
\end{itemize}

\subsection{Interval extension}
Assuming the availability of a correctly rounded function $f$, we want to obtain a correctly rounded inclusion function for $f$, that is, an interval-valued function $\inclusion{f}$ such that the endpoints of the resulting interval are correctly rounded outward.

TIGHT's interval class wraps the NFG interval library \cite{nfg}, which guarantees that the result of every operation contains the true result, thanks to conservative outward rounding. Specifically, the library is initialized by setting the FPU rounding mode towards positive infinity, and the lower end of the interval is represented internally with opposite sign, which results in downward rounding without changing the rounding mode.

An issue that must be considered is how to deal with input intervals containing points outside the domain of $f$. We discuss this in Section \ref{}, and until them, we limit our discussion to identifying such ill-posed inputs.

The challenge of extending a function to intervals with correct rounding is twofold.
First, we need an expression for the range of the function; this is obtained by enumerating the possible cases for a given function.
Then, these expressions must be instantiated on the input datum and rounded correctly - downward for the lower bound and upward for the upper one.

\subsubsection{Odd and monotonic functions}
When $f$ is monotonically increasing on $[\intlo{x}, \inthi{x}]$, the range of the function on an interval is easily obtained as $\inclusion{f}([\intlo{x}, \inthi{x}]) = [f(\intlo{x}), f(\inthi{x})]$; if it is monotonically decreasing, the two endpoints are swapped.
If $f$ is not monotonic on $[\intlo{x}, \inthi{x}]$, we need to know where $f$ attains its extrema on the interval. As we will see for some functions, even deciding that $[\intlo{x}, \inthi{x}]$ lies in a part of the domain where $f$ is monotonic is tricky in floating point arithmetic.

Once we know how to compute the range of the function in exact arithmetic, correct rounding amounts to rounding the left endpoint down, and the right endpoint up. Given that we are operating in round-upward mode, the latter is free.
To round down we could change the rounding mode and reset it after the operation, but changing roundind modes flushes the CPU pipeline, thus it is an expensive operation that fortunately can be avoided in several cases.
For example, when $f$ is odd, the result of $f(x)$ rounded down is easily computed in upward rounding mode as $-f(-x)$, since negation is an exact operation. Several elementary functions are odd, so this property is helpful for our purposes.

For a monotonically increasing and odd function $f$ for which we have access to a correctly rounded implementation, the correctly rounded $\inclusion{f}$ is $\inclusion{f}([\intlo{x}, \inthi{x}]) = [-f(-\intlo{x}), f(\inthi{x})]$. Fortunately, many of the functions of interest to us satisfy both these properties:
\begin{itemize}
	\item arcsine and arctangent
	\item hyperbolic sine and tangent
	\item hyperbolic arcsine and arctangent
	\item cube root
	\item error function
\end{itemize}

\subsubsection{Inverse trigonometric, exponential and logarithmic functions}
When we drop the oddity assumption, we face the problem of how to round down the result of a call to the correctly-rounded library function without changing the rounding mode, since doing so might degrade performance.
An easy way to deal with this is to compute the result with upward rounding, then correcting this by taking the immediately preceding FP value.
However, if the exact result of the function is representable as a floating point number, correct rounding imposes that the exact result will be returned; in these special cases, if use the immediately smaller FP value, we get an error of 1 ULP.
To address this, we would like to only perform this operation if the upward rounding happened. Unfortunately, while the IEEE754 standard does provide a way to check this via status flags, CORE-MATH currently does not guarantee that status flags are set correctly.
For many functions it is possible, however, to check \emph{a priori} if the result will be exactly representable as a floating point number, and round down when it is not.

\FS{\dots}

\subsection{Points outside the domain}

%\section{Results and comparison}
\begin{table}
\centering
\label{table:benchmarks}
\caption{\FS{add a caption}}
\begin{tabular}{r|rr|rr}
\multirow{2}{*}{Operation} & \multicolumn{2}{c|}{Avg. ns} & \multicolumn{2}{c}{width in ULPs} \\
& Filib++ & TIGHT & Filib++ & TIGHT \\
\hline
\texttt{+} &  &  &  &  \\
\texttt{-} &  &  &  &  \\
\texttt{*} &  &  &  &  \\
\texttt{/} &  &  &  &  \\
\texttt{abs} &  &  &  &  \\
\texttt{sqr} &  &  &  &  \\
\texttt{sqrt} &  &  &  &  \\
\texttt{rsqrt} &  &  &  &  \\
\texttt{cbrt} &  &  &  &  \\
\texttt{sin} &  &  &  &  \\
\texttt{cos} &  &  &  &  \\
\texttt{tan} &  &  &  &  \\
\texttt{asin} &  &  &  &  \\
\texttt{acos} &  &  &  &  \\
\texttt{atan} &  &  &  &  \\
\texttt{sinpi} &  &  &  &  \\
\texttt{cospi} &  &  &  &  \\
\texttt{tanpi} &  &  &  &  \\
\texttt{asinpi} &  &  &  &  \\
\texttt{acospi} &  &  &  &  \\
\texttt{atanpi} &  &  &  &  \\
\texttt{sinh} &  &  &  &  \\
\texttt{cosh} &  &  &  &  \\
\texttt{tanh} &  &  &  &  \\
\texttt{asinh} &  &  &  &  \\
\texttt{acosh} &  &  &  &  \\
\texttt{atanh} &  &  &  &  \\
\texttt{exp} &  &  &  &  \\
\texttt{exp2} &  &  &  &  \\
\texttt{exp10} &  &  &  &  \\
\texttt{expm1} &  &  &  &  \\
\texttt{exp2m1} &  &  &  &  \\
\texttt{exp10m1} &  &  &  &  \\
\texttt{log} &  &  &  &  \\
\texttt{log2} &  &  &  &  \\
\texttt{log10} &  &  &  &  \\
\texttt{log1p} &  &  &  &  \\
\texttt{log2p1} &  &  &  &  \\
\texttt{log10p1} &  &  &  &  \\
\texttt{erf} &  &  &  &  \\
\texttt{erfc} &  &  &  &  \\
\texttt{pow} &  &  &  &  \\
\texttt{hypot} &  &  &  &  \\
\texttt{atan2} &  &  &  &  \\
\end{tabular}
\end{table}


%% !TEX root = STAG25-MiSo2.tex

\section{Conclusions and future work}
We have introduced TIGHT, an efficient C++ library for correctly rounded interval arithmetic built on top of the state-of-the art in both correctly rounded computations and interval arithmetic.

TIGHT is designed to be future-proof: if correctly rounded mathematical routines become the standard in the coming years, or more performant CR methodologies are developed, switching the underlying library requires minimal changes.
Crucially, since the result of a CR operation is well defined, TIGHT will continue to return the same results \emph{forever} on all machines, even in the event of a library change
%(the only exception to this is, of course bug fixes)
.

The library can still be improved in several ways:
\begin{enumerate}
	\item NFG's arithmetic operations owe their speed to vectorization. Vectorized mathematical libraries exist \cite{sleef} but we are not aware of a correctly-rounded solution. In any case, having non-CR, faster interval routines could still be useful for applications that can cope with error and inconsistencies across machines.
	\item TIGHT currently does not conform with the IEEE 1788-2015 standard for interval arithmetic \cite{ieee1788}. Adding compliance and testing with a framework such as \cite{Benet23} would make the library more easily integrated. This includes adding conformant support for infinity and NaN values.
	\item As mentioned in Section \ref{sub:sincos}, integrating CR functions for multiplication with an irrational constant value as in \cite{Brisebarre:2005} would bring a more consistent speed for $\sin$ and $\cos$ on arbitrary arguments.
	\item CORE-MATH also contains a CR implementation of the gamma function (included in the C++ standard library), that could be used to build an interval implementation, though it is not clear whether this would be useful in practice.
\end{enumerate}



%-------------------------------------------------------------------------
% bibtex
\bibliographystyle{eg-alpha-doi}  
\bibliography{biblio}        


\end{document}
